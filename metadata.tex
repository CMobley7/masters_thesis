\def\myabstract{This paper will present a multistage localization approach for
an autonomous industrial mobile manipulator (AIMM). This approach allows tasks with
an operational scope outside the range of the robot’s manipulator to be completed
without having to recalibrate the positon of the end-effector each time the robot’s
mobile base moves to another position. This is achieved through localizing the AIMM
within its area of operation (AO) using adaptive Monte Carlo localization (AMCL),
which relies on the fused odometry and sensor messages published by the robot, as
well as a 2-D map of the AO, which is generated using an optimization-based smoothing
simultaneous localization and mapping (SLAM) technique. The robot navigates to a
predefined start location in the map incorporating obstacle avoidance through the
use of a technique called trajectory rollout. Once there, the robot uses its RGB-D
sensor to localize an augmented reality (AR) tag in the map frame. Once localized,
the identity and the 3-D position and orientation, collectively known as pose, of
the tag are used to generate a list of initial feature points and their locations
based on \textit{a priori} knowledge. After the end-effector moves to the approximate
location of a feature point provided by the AR tag localization, the feature point’s
location, as well as the end-effector’s pose are refined to within a user specified
tolerance through the use of a control loop, which utilizes images from a calibrated
machine vision camera and a laser pointer, simulating stereo vision, to localize
the feature point in 3-D space using computer vision. This approach was implemented
on two different ROS enabled robots, Clearpath Robotics’ Husky and Fetch Robotics’
Fetch, in order to show the utility of the multistage localization approach in executing
two tasks which are prevalent in both manufacturing and construction: drilling and
sealant application. The proposed approach was able to achieve an average accuracy
of $\pm$ 1 mm in these operations, verifying it’s efficacy for tasks which have a
larger operational scope than that of the range of the AIMM’s manipulator and its
robustness to general applications in manufacturing.}
\def\myacknowledgements{Insert Acknowledgement}
\def\mykeywords{Autonomous Navigation, SLAM, Visual Servoing, Mobile Manipulation, Computer Vision, State Machine}
\def\mycommittee{Tomonari Furukawa\\ Brian Lattimer\\ Kevin Kochersberger\\}
\def\mydate{\date}

%To change linespacing, uncomment the next line
%\def\mylinespace{\onehalfspacing}
